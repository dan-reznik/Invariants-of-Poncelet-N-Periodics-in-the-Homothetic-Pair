In this section we define polygons,
their moduli spaces,
and various classes of functions on these spaces
(polygonomials, ratiogonal functions, etc).

%   polygonomial  = polynomial on polygon space, (+ G-invariant?)
%   ratiogonal    = ratio of polygonomials

Let $F$ be a principal homogeneous space (a torsor) over a group $E$.
In our main examples either $F$ will be an affine space over a vector space $E$,
of $F$ will be a torsor over a semi-abelian variety $E$
(e.g. $F$ will be a curve of arithmetic genus one and $E$ will be its generalized Jacobian),
but so far any $E$ (even non-abelian) would suffice,
however we will use additive notation $(+,-,0)$ for the operations on $E$
and action $E:F$ to reserve the multiplicative notation for the action
of another group $G$ to be defined shortly later.


\begin{definition} \label{d:marked-f-polygon}
For $n\geq 0$ define \emph{a marked ($F$-)polygon}
to be a map $P : \Z/n \to F$,
so that $P(i), i \in \Z/n$ is a set of points in $F$,
that we interpret as the vertices.
The number $n$ will be called
\emph{gonality}
\footnote{A relation with gonality of algebraic curves will be explained shortly.},
so $P$ will be called an $n$-gon.
\footnote{Except for the case $n=0$ when we call it an $\infty$-gon
(abusing notation, gonality zero is the same as gonality infinity).
Note that so far it would suffice to consider only $\infty$-gons
and defining $n$-gons as $\infty$-gons with the map $P$ being $n$-periodic.}
\end{definition}

One constructs \emph{oriented sides} $e(i) \in E$ as the (right) differences

\begin{equation} \label{eq:P2e}
P(i) = P(i-1) + e(i)
\end{equation}

For $n>0$ the vectors $e(1),...,e(n)$ satisfy a relation

\begin{equation} \label{eq:zerosum}
  e(1) + e(2) + ... + e(n) = 0       
\end{equation}

Two polygons $Q,P$ have equal associated side-vectors
if and only if they differ by a (left) translation,
i.e. exists a $t \in E$ such that for all $i$
\begin{equation} \label{eq:left-translation}
Q(i) = t + P(i)
\end{equation}

The polygons that differ by a (left) translation
are called ($E$-)equivalent,
and the space of the ($F$-)polygons up to a (left) translation
is the (left) quotient $E \backslash F^n$.

\begin{definition} \label{d:marked-e-polygon}
\emph{A marked $E$-polygon}
is a collection of $n$ elements of $E$ subject
to the condition \eqref{eq:zerosum}.
\end{definition}

\begin{remark} \label{rem:twist}
Analogously one defines for any $τ\in E$ the \emph{$τ$-twisted ($E$-)polygons}
by replacing $0$ in the right hand side of \Cref{eq:zerosum} by $τ$.
Similarly, \emph{$τ$-twisted ($F$-)polygons} are given by
quasi-periodic functions $P : \Z \to F$, i.e.
\begin{equation} \label{eq:twisted-f-polygon}
P(i) = P(i-n) + τ
\end{equation}
\end{remark}

By a convenient abuse of notation
(that happens only for non-trivial $E$-torsors $F$)
we make no further distinction between $E$-equivalence classes of $F$-polygons
and $E$-polygons.

\begin{definition} \label{d:poncelet-polygon}
A marked/oriented/unoriented ($E$- or $F$-)polygon is called
\emph{Poncelet ($E$- or $F$-)polygon}
if the function $e$ is constant. Clearly Poncelet $\infty$-gons
are classified by vectors of $E$, and Poncelet $n$-gons for $n\geq 1$
are classified by $n$-torsion vectors $e \in E[n]$.
\end{definition}

Note that the cyclic group acts on the space of the marked polygons by shifting the marking
$s P(i) = P(i+1)$, and later (after setting up the moduli space of marked polygons)
by an \emph{oriented polygon} we would mean a $\Z$-equivalence class of the polygons.
Similarly an \emph{unoriented polygon} is an equivalence class of polygons with respect to the dihedral group,
generated by the shifts of the cyclic group and a reflection $r P(i) = P(-i)$.

For $0<n<\infty$ given any $(n-1)$ vector-sides $e(i)$
the remaining one is uniquely determined by \eqref{eq:zerosum},
thus the space of all marked $n$-gons in $E$ can be identified
with $(n-1)$-st Cartesian power of $E$,
denoted as $E^{\times (n-1)}$ or $E^{n-1}$.
Consider the space of (polynomial/rational) functions on $E^{n-1}$.

Further endow $E$ with an action of a group $G$ by group homomorphisms of $E$.
\begin{example} \label{ex:G:E}
\begin{itemize}
\item $E$ is a vector space $V$ with a symmetric bilinear form $B$ and $G = \SO(V,B)$
a special orthogonal group
\item $E$ is a complex vector space $V$ with a sesquilinear form $B$ and $G = \Un(V,B)$
a unitary group
\item $E=V$ is same as above, but group $G$ is extended by homotheties $v\mapsto λ v$
% \item $E$ an elliptic curve and $G$ generated by involution $e\mapsto -e$
\end{itemize}
\end{example}

The group $G$ acts (diagonally) on the spaces of polygons: for all $g \in G$
\begin{equation} \label{eq:action}
(e(1),...,e(n)) = e \mapsto  g e := (g e(1), ..., g e(n))
\end{equation}

and gives an equivalence relation thereon ($G$-equivalence).
In the case of twisted polygons (\cref{rem:twist}) the group $G$ also acts on
the twists $t$.

We would like to "identify" $G$-equivalent polygons in order to consider
moduli spaces of $G:E$-polygons, whose points are (some) orbits of this action.

Thus polynomial/rational functions on the moduli space of $G:E$-$n$-gons are defined
as $G$-invariant polynomial/rational functions on $E^{n-1}$.

Let us call them ($G$-)polygonomials and ($G$-)ratiogonal functions, respectively.

The case $n=2$ corresponds to $G$-invariant functions on $E$,
and in our situation typically there won't be that many non-constant invariants:
\begin{example} \label{ex:B11}
In \Cref{ex:G:E} the ring of invariants with respect to $\SO$ or $\Un$
is $\k[B_{11}]$, the ring of polynomials freely generated by $B_{11} := B(e_1,e_1)$.
The generator is not invariant with respect to the homotheties,
but has grading $2$.
\end{example}
In what follows let us localize to $e(i)\neq0$, so that
$G$-invariants $B_{ii}$ from example \Cref{ex:B11} are non-zero/invertible.

The extra grading with respect to the homotheties is useful,
and typically for geometric invariant theory, we will consider a bigger group
$G^* \supset G$ and its character $χ : G^* \to \C^*$
such that $G = \Ker χ$. Then instead of $G$-invariants we consider
quasi-invariants with respect to tensor powers of $χ$:
\begin{equation}
\oplus_{k\in\Z} \k[E^{n-1}]^{kχ}
\end{equation}

Every map $f: E^{n-1}\to E$ can be used to pull back the quasi-invariants
$f^*: \k[E]^{kχ} \to \k[E^{n-1}]^{kχ}$.

\begin{example}
In \Cref{ex:G:E} natural $G$-invariants on $E^2$ are tautologically
given by the pairings $B_{ij} := B(e_i,e_j)$ for $i,j\in \Z/n$.
Functions $B_{ij}$ satisfy linear relations
\begin{equation}
\sum_i B_{ij} = 0 = \sum_j B_{ij}
\end{equation}
and $B_{ij} = B_{ji}$ (resp. $B_{ij} = \ol{B_{ji}}$),
so only $\frac{n(n-1)}{2}$ of them are linearly independent.
\end{example}

Let us see the $G$-invariants of $E$-triangles
(equivalently, $G$-invariants of a pair of vectors in $E$).

\begin{example}
There are three quasi-invariants with character $χ$,
and the field of invariants is freely generated by their two ratios.
Indeed, if $e_2$ lies in a complex line spanned by non-zero $e_1$
there is a unique element of $\tilde{G}$ that maps $e_1$ to $1\in \C$.
Then the real and imaginary coordinates of the image of $e_2$ (i.e. of $e_2/e_1$)
are the free generators of the field of invariants.
Their ratio is the (co)tangent of the angle between vector-sides $e_1$ and $e_2$
of the triangle, multiplied by $-1$.
\end{example}


