\section{A fazer.}

Em matemática
as demonstrações são posteriores às declarações,
e as declarações são posteriores às definições.
O círculo virtuoso é:
 definições, declarações, demonstrações, definições, ...

Um problema principal com este trabalho é que as demonstrações
são triviais, mas é muito mais difícil
declarar o que está acontecendo
e fazer as definições naturais.

Algumas coisas a fazer:
\begin{enumerate}
\item Escreva a seção sobre o modelo de Poncelet, ou não use a terminologia indefinida.
\item Define as coisas.
 \begin{itemize}
 \item ''Dan-invariant'' = invariant with respect to the right rotation (one made before the scaling).
 \item ''Dan simple geometric'' = invariant with respect to the left rotation (one made after the scaling).
 \end{itemize}
In other words,
 on the group of affine transformations 
\begin{equation} \label{eq:affine-group}
0 \to \k^2 \to Aff(2) \to \GL(2,\k) \to 1
\end{equation}
we have two copies of the group of (oriented) Euclidean transformations
\begin{equation} \label{eq:euclidean-group}
0 \to \k^2 \to Euc(2) \to \SO(2,\k) \to 1
\end{equation}
acting, one from the left, another from the right, by
\begin{equation}
(L,R) : T \mapsto L T R^{-1}
\end{equation}
 A quantity is ''Dan simple geometric invariant''
  if it is invariant with respect to both of these actions.
\item Também precisamos os nomes mais razoáveis para estes termos,
mas compatíveis com a nomenclatura matemática dominante.
\item Explain that the vectors of coordinate and momentum are invariant with respect the the right rotation,
but \emph{covariant}/\emph{equivariant} with respect to the left rotation.
\item Explain how to cook up invariants from covariants using invariant forms (such as the area and the scalar product).
\end{enumerate}

\section{Area orientada como uma função.}

\eqref{:as}: the signed area of the evolute polygon (except squares).
\textcolor{red}{sergey: area is not associated with vertices, do you want to say area elements / determinants?}
{Mudei a ordem dos itens exatamente pela razão que a area na primeira vista parece não ser uma função. Mas isso é só na primeira vista.

Há dois jeitos como podemos fazer sentido da area como uma função de $P$:

\begin{itemize}
\item mais specifico: dizer que a area orientada de polígono é soma das areas orientadas dos triangulos.
Expliquei este depois quando expandiu a subseção sobre areas.
Então podemos considerar a area do triangulo $POρ(P)$ como uma função.

\item mais geral: por qualquer $F: X\to Y$ temos $F_* F^* F_* = F_*(1)\cdot F_*$,
então podemos pensar sobre area do polígono como uma função $F^* F_* f$ onde $f$ é a area do triângulo.
Isso é, se já sabemos que $ρ^{\circ N} = Id$,
por qualquer ponto $P$ podemos considerar o $N$-gono orientado com vertices $ρ^{\circ k} P$,
e a area dele é uma função de $P$.
\end{itemize}

Ainda não escrevi nenhum destes dois explanações, mas precisamos explicar em um ou outro jeito o fato que
a area é uma função.
}
