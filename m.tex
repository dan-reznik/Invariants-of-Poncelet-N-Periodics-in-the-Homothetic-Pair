% arara: lualatex
% arara: lualatex
\RequirePackage[l2tabu,orthodox]{nag}   % nag about obsolete LaTeX
\documentclass[12pt]{article}
%\usepackage{uni8}
\usepackage{geometry}
%\geometry{paperwidth=441pt, paperheight=666pt}
\usepackage{iftex}                      % to see which engine is compiling this file
\iftutex
 \usepackage{fontspec}                  % font selection in xelatex/lualatex
 \defaultfontfeatures{Ligatures=TeX}
 %\setmonofont{inconsolata}
 \setmainfont{TeX Gyre Termes}
 \usepackage{unicode-math}             % !\sigma doesn't show!  amssymb,amsmath equiv.
 \setmathfont{Stix Two Math}
 \setmonofont{Inconsolata}
 %\usepackage{polyglossia}
 %\setdefaultlanguage{english}
 \ifLuaTeX
 %  \usepackage{luatex85}                % <= luatex backward compatibility
   \usepackage{lualatex-math}
 \fi
\else
 % \usepackage[utf8]{inputenc}
 \usepackage[T1,T2A]{fontenc}
 \usepackage{alphabeta}
 \usepackage{amssymb}
 \usepackage{amsmath}
 %\usepackage{newtxtext,newtxmath}
\fi

\usepackage{amsthm}
\usepackage{csquotes}
%\usepackage[french,portuges,main=english]{babel}
\usepackage[english]{babel}
\usepackage[unicode]{hyperref}
\usepackage[dvipsnames]{xcolor}
%\hypersetup{backref}
\hypersetup{colorlinks, bookmarksopen=false} %,bookmarksnumbered,pdfstartview=FitW}
\hypersetup{pdfstartview=}
\hypersetup{
    linkcolor={red!50!black},
    citecolor={blue!50!black},
    urlcolor={blue!80!black}
}
\usepackage{night}
\usepackage[mark]{gitinfo2}
\usepackage{cleveref}                    % load _after_ hyperref. Usage: "in \Cref{t:1}"
\usepackage[maxbibnames=99,safeinputenc]{biblatex}


\numberwithin{equation}{section}

\newtheorem{dummy}{dummy}[section]           % s: \label{s:sectionlabel}

\newtheorem{corollary}[dummy]{Corollary}     % c: \label{c:primer}
\newtheorem{definition}[dummy]{Definition}   % d: \label{d:imya}
\newtheorem{example}[dummy]{Example}         % e:
\newtheorem{lemma}[dummy]{Lemma}             % l:
\newtheorem{proposition}[dummy]{Proposition} % p:
\newtheorem{remark}[dummy]{Remark}           % r:
\newtheorem{theorem}[dummy]{Theorem}         % t:
% \(Theorem\|Corollary\|Lemma\|Proposition\|Remark\|Example\|Conjecture\|Definition\)
%\newtheorem{conjecture}[dummy]{Conjecture}

\newcommand\A{\mathbf{A}}                    % [A]ffine space. Almost not used.
\renewcommand\C{\mathbf{C}}                  % [C]omplex numbers
\newcommand\F{\mathbf{F}}                    % [F]inite [F]ield
\renewcommand\P{\mathbf{P}}                  % [P]rojective spaces
\newcommand\Q{\mathbf{Q}}                    % Rational numbers
\newcommand\R{\mathbf{R}}                    % [R]eal numbers
\newcommand\N{\mathbf{N}}                    % [N]atural numbers
\newcommand\T{\mathbf{T}}                    % [T]orus - a circle
\newcommand\Z{\mathbf{Z}}                    % Integer [Z]ahlen
\renewcommand\k{\mathbf{k}}                    % Some base field (R,C,Q,F_q,etc)
\DeclareMathOperator\Jac{Jac}		     % Jacobian
\DeclareMathOperator\Pic{Pic}                % a.k.a. Picard variety
\providecommand {\arxiv}  [1] {\href{https://arxiv.org/abs/#1}{\texttt{arXiv:#1}}}
\addbibresource{m.bib}
\begin{document}
%\ttfamily
\begin{abstract}
\Cref{sec:concentric,sec:cotangent} make a proof of conservacy of
the sum of cotangents, so these are to be included in the paper.
\Cref{sec:harmonic} gives three equivalent re-formulations of conservacy
for the periodic cases.
The rest is bla-bla or some explanations.
\end{abstract}
%\maketitle
\tableofcontents 

\section{Poncelet Geometry}

Define configuration space $X$ associated with a pair of conics.
Discuss general case when it is a smooth genus one curve
and degenerate cases, where the curve becomes singular
or even reducible (like in the case of two concentric circles).
Discuss degeneration of the group law.

Discuss moduli space of configurations.

\section{Poncelet Dynamics}

Define two involutions and their composition --- Poncelet map $T : X$.
Recall Poncelet porism.
Explain periodic/aperiodic dichotomy in terms of torsion/non-torsion in the group.
In the periodic case $\frac{α}{2π} = \frac{w}{N} \in\ (-\frac12,\frac12)$ where $N\in\N$ is the period and $w\in\Z$ is the winding number.
Discuss torsion configurations as subset in moduli.
Explain that both torsion and non-torsion are dense.


\section{Rigidity lemma}

In one of their proofs
\cite{AST} use Liouville's theorem from complex analysis:
a bounded holomorphic entire function is constant.
A weaker analogue of this in algebraic geometry says that any regular function
on a projective line is constant, or more generally,
any regular map from a connected projective variety to an affine variety is constant.
Based on this one proves a Rigidity Lemma (cite e.g. Mumford):
for any regular map $f: B\times X\to Y$ with $X$ proper and $B$ connected,
if $f(b_0,X)$ is constant for one value $b_0\in B$ then $f(b,x) = g(b)$ for some $g: B\to Y$.

Explain non-example, and what have to be checked in order to apply it.

\section{Galois action}
Particular values of the average for $α\in2π\Q$ give various interesting formulae in the (neighbours)
of the cyclotomic field $\Q(ζ_N)$.
To expand my e-mail.

\section{Ergodic theorem}

We refer to \cite{Moore2015} for a brief history of ergodic hypothesis
and the ergodic theorems
proved/refined in 1931--1933
by John von Neumann \cite{Neumann1932},
George Birkhoff \cite{Birkhoff1931a,Birkhoff1931b}
and Khinchin.

Ergodic theorem says that for ergodic maps $T: X$ on measurable space $X$
and nice functions $f: X\to \R$ the averages over time equal to the averages over space $X$.

When $α$ does not belong to $2π\Q$ the translation $T_α (β) := β+α \mod 2π$ acts on the circle ergodically,
so we can apply it to deduce that time-average cotangent equals to the space average that we computed in \Cref{space-average}.

For $α\in2π\Q$ (period cases) one \emph{can not} just deduce the same formula by limiting 
--- the time-averages are well-defined \emph{but a priori not uniformly continuous}.
A notion of strong ergodicity serves to provide a uniform convergence, but I cannot see how to apply it in our case.
Various counter-examples show it might be not such a simple problem.

\section{Periodic averages and time averages}

Let $f$ be a function on $X$ with values in an abelian group $A$.

If $T: X$ is an endomorphism of period $N$ (i.e. $T^{\circ N} := T\circ T\circ\dots\circ T = Id_X$) consider
\[ \sum_{k=1}^N f(T^{\circ k} x) \]
This is pretty well defined in many contexts, but for application of the ergodic theorem it is more convenient
to consider the case where group $A$ is uniquely $N$-divisible, e.g. a field of characteristic $0$ (or more than $N$).
Then periodic average is obtained by division of the sum by $N$
\[ \frac{1}{N} \sum_{k=1}^N f(T^{\circ k} x) \]

Generalization
\begin{definition}
If $T$ not necessarily has period $N$ one can nevertheless consider a sequence
\[ \frac{1}{n} \sum_{k=0}^{n-1} f(T^{\circ k} x) \]
and look for its limit, the so-called time average
\[ \bar{f}(x) := \lim_{n\to\infty} \frac{1}{n} \sum_{k=0}^{n-1} f(T^{\circ k} x) \]
\end{definition}

\begin{proposition}
If $T$ is $N$-periodic, 
then for every $x$ for which the periodic average of $f$ is defined,
the time-average limit exists and equals to the periodic average.
\end{proposition}

\section{Harmonic analysis}
\label{sec:harmonic}

A square-integrable function $f$ on a circle $\T = \R/\Z$.
can be represented as a periodic function on $\R$ 
\[ f(x+1)=f(x) \]
with Fourier expansion
\[ f(x) = \sum_{n\in\Z}   a(n)   q^n \]
where $n$ runs over integers,
\[ q := \exp(2πi\cdot x) \]

and $a(n)$ are Fourier coefficients (complex numbers), 
\[ a(-n) = \int_{x=0}^1 f(x) q^n dx \]

For a fixed $y\in \T$
consider the operator $U$

\[ (Uf) (x) = f(x+y) \]

Define
\[ g_N := (1+U+U^2+...+U^{N-1}) f \]
explicitly
\[ g_N(x) = f(x) + f(x+y) + f(x+2y) + ... + f(x + (N-1)y) \]

By geometric progression
\[ (1-U^N)=  (1+U+U^2+...+U^{N-1}) (1-U) \]

Assume $U^N=1$  i.e. $y=\frac{w}{N}$  for an integer $w \in (-N/2,N/2]$.
Moreover assume $U^M \neq 1$ for $M<N$, i.e. $w$ is coprime to $N$.
Under this assumptions
\[  ζ := \exp(2πi\cdot y) \]
is the primitive N-th root of unity, 
so $(1-ζ^n) = 0$ iff $n$ is divisible by $N$ (I will shorten this to $n\in(N)$).

\begin{proposition}
The following are equivalent
\begin{enumerate}
\item $g_N = 0$
\item exists $h$ such that $f(x) = h(x) - h(x+y)$
\item $a(n)=0$ for all $n$ divisible by $N$
\end{enumerate}
\end{proposition}
\begin{proof}
$(2)\implies(1)$:   
If $h$ exists, then $g = (1+U+U^2+\dots+U^{N-1}) (1-U) h =  (1-U^N) h = 0 h = 0$,
so the image of the operator $(1-U)$ is inside 
the kernel of the operator $(1+U+U^2+\dots+U^{N-1})$.

More generally, the operator $U$ is diagonal in the orthonormal base $\{q^n\}_{n\in\Z}$.
It multiplies $n$-th Fourier coefficient by $ζ^n$.
For any polynomial $P$ the operator $P(U)$ multiplies $n$-th coefficient by $P(ζ^n)$.
Thus the image of $(1-U)$ consists of functions 
$\sum_{n\in\Z} b(n) q^n$  with $b(n)=0$ for $n\in(N)$.
Similarly, the kernel of the operator $(1+U+U^2+\dots+U^{N-1})$
consists of the functions $\sum_{n\in\Z} c(n) q^n$
such that for any $n$
\[ c(n) \cdot  (1 + ζ^n + ζ^{2n} + \dots + ζ^{(N-1)n}) = 0 \]
use that $(1 + ζ^n + ζ^{2n} + \dots + ζ^{(N-1)n}) \neq 0 \iff  n\in(N)$:
if $n\in(N)$ then $ζ^{kn}=1$ for all $k\in\Z$, so the sum equals $N$,
if $n$ is not in $(N)$ then $(1-ζ^n)$ is invertible
and $(1-ζ^n)((1 + ζ^n + ζ^{2n} + \dots + ζ^{(N-1)n}) = 1-ζ^{Nn} = 0$
implies $1 + ζ^n + ζ^{2n} + \dots + ζ^{(N-1)n} = 0$.
So the kernel of the operator $1+U+\dots+U^{N-1}$ 
is made of the sums $\sum_{n\in\Z} c(n) q^n$ 
with $c(n)=0$ for $n\in(N)$,
the same as the image of the operator $(1-U)$.
\end{proof}

It suggests that if there is an experimentally conserved quantity $F$ 
then it makes sense to see if $F'$ (the derivative) 
is representable as a difference $h(x) - h(Tx)$, 
where $T$ is the Poncelet map.
Here we use derivative just to ensure that the integral of $f$ over circle is zero,
so $g_N=0$ but $G_N$ is a constant ($N \int F dx$).

This suggests that for every conserved quantity we need to take a closer look into its Fourier expansion in terms of the natural parameter.

The natural parameter $β$ for 
(the universal cover of) 
(the elliptic double cover of) 
the Poncelet conic is defined by the property
\[ β(T Q) - β(Q) \]
being constant (independent of $Q$).

\subsection{Hilbert transform on the circle}
Cotangent is a kernel for the Hilbert transform on the circle.
Also its expansion into the infinite sum over poles
plays the fundamental role in establishing modularity
of Eisenstein series, hence can be thought as a fundation for modular forms.

\section{Ronaldo formulae}

\begin{proposition}\label{prop:somacotan}
\[ \frac{1}{n} Cot_n =
\frac{1}{n} \sum_{k=1}^n \cot_k =  - \cot \left({\frac{2π}{n}}\right) \cdot \frac {\left( {a}^{2}+{b}^{2} \right) }{2ab}
\] 
\end{proposition}


\begin{proposition}\label{prop:somaL2}
\[ \frac{1}{n} L_n = \frac{1}{n} \sum_{k=1}^n L_k^2 = \left( 1-\cos \left(  {\frac {2π}{n}} \right)  \right)  \left( 
{a}^{2}+{b}^{2} \right) 
\]
\end{proposition}

\begin{proposition}\label{prop:areaN}
\[\frac{1}{n} A_n= \sin \left( {\frac{2π}{n}} \right) \frac{ab}{2}\]
\end{proposition}

\begin{corollary}
\[ 4\left(1-\cos\left(\frac{2π}{n}\right) \right) \frac{A_n}{n} \frac{Cot_n}{n}  +  \cos\left(\frac{2π}{n}\right) \frac{L_n}{n} = 0\]
\end{corollary}
\begin{proposition}\label{prop:somacotan2}

Let \[ c_n=\cos\left(\frac{\pi}{n}\right), \;\;\;\;\;  s_n=\sin\left(\frac{\pi}{n}\right). \]
Then,
\begin{align*}
Cot2_n=\sum_{k=1}^n \cot_k^2 = &\frac {n \left(  \left( {a}^{2}+{b}^{2} \right) ^{2} (c_n^{
4}-  c_n^{2}) +\frac{3}{8}\,{a}^{4}+\frac{1}{4}
\,{a}^{2}{b}^{2}+\frac{3}{8}\,{b}^{4} \right) }{4{a}^{2}{b}^{2} s_n^{2}
 c_n^{2}}\\
 =& -\frac{((a^2+b^2)^2\cos(4\pi/n)+2 (a^4+ b^4)) n}{    4 (\cos(4\pi/n)-1)  a^2 b^2)}\\
 =&  \frac{\left( 3\,a^{4}+2\,a^{2}b^{2}+3\,b^{4}\right) Cot_n^2}{2n\left( a^{2}+b^{2} \right)^{2}}+ \frac{
 \left(a^2-b^2 \right)^{2}   n}{8a^{2}b^{2}}
\end{align*} 
\end{proposition}

\[ \lim_{n\to\infty}\frac{Cot_n}{n^2}=-\frac{ a^2+b^2}{4\pi ab},\;\;\;\lim_{n\to\infty}\frac{Cot2_n}{n^3}= \frac{ a^4+2 a^2 b^2 +3 b^4}{ 32\pi^2 a^2b^2   } \]


Generalization: 
replace the rational angle $\frac{2π}{n}$ 
by arbitrary angle $α$ 
and the periodic averages 
$\frac{Cot_n}{n}$, $\frac{A_n}{n}$, $\frac{L_n}{n}$ 
by the respective time-averages.


\section{Concentric case}
\label{sec:concentric}

Rotation by angle $α$ is given by
\[ z\to z' = e^{i α} \cdot z \]
Since $e^{i α} = \cos α + i \sin α$ for $z = x + i y$ and $z'=x'+iy'$ we have
\[ x'+iy' = z' = e^{i α} \cdot z = (\cos α + i \sin α) (x + i y) = (x \cos α - y \sin α) + i (x \sin α + y \cos α) \] 
or
\begin{align*} x' &= (x \cos α - y \sin α)    &     y' &= (x \sin α + y \cos α) \end{align*}
Also if $z = e^{i β}$ then $z' = e^{i (β+α)}$ and respectively $x'=\cos(β+α), y'=\sin(β+a)$.
So explicitly we can write down coordinates for the next and the previous points using $\cos(-α)=\cos(α)$ and $\sin(-α)=-\sin(α)$.
If $T (x_i,y_i) = (x_{i+1},y_{i+1})$a then
\begin{align} 
x_{i+1} &= (x_i \cos α - y_i \sin α)    &     y_{i+1} &= (x_i \sin α + y_i \cos α)  \\
x_{i-1} &= (x_i \cos α + y_i \sin α)    &     y_{i-1} &= (-x_i \sin α + y_i \cos α)  
\end{align}

And if $X_i = a x_i$ then

\begin{align} \label{scaled-dynamics} 
X_{i+1} &= a(x_i \cos α - y_i \sin α)   &    y_{i+1} &= ( x_i \sin α + y_i \cos α)  \\
X_i     &= a x_i                        &    y_i     &=                y_i          \\
X_{i-1} &= a(x_i \cos α + y_i \sin α)   &    y_{i-1} &= (-x_i \sin α + y_i \cos α)  
\end{align}

\section{Cotangents}
\label{sec:cotangent}

Great simplification of cotangents.
Cotangent of angle $ABC$ is a rational function of coordinates of the vertices.
Explicitly, if $B=O:=(0,0)$ and $C=H:=(*,0)$, $A=(x,y)$ then
\[ \cot = \frac{x}{y} \]
If $C=(x_1,y_1)$ and $A=(x_3,y_3)$ consider the associated complex numbers 
\begin{align*} 
c &= x_1 + i y_1 & 
a &= x_3 + i y_3
\end{align*}
Their ratio equals
\[ d = \frac{c}{a} 
 = \frac{x_1 + i y_1}{x_3 + i y_3} 
 = \frac{(x_1 + i y_1)(x_3 - i y_3)}{(x_3 + i y_3)(x_3 - i y_3)} 
 = \frac{(x_1 x_3 + y_1 y_3) + i (y_1 x_3 - y_3 x_1)}{x_3^2 + y_3^2}
\]
Angle $AOC$ is homothetic to angle $HOD$, hence
\begin{equation} \label{cotangent}
\cot = \frac{x_1 x_3 + y_1 y_3}{y_1 x_3 - y_3 x_1} 
\end{equation}
Formula for the cotangent of abstract triple is obtained by a shift, but it will be easier to do shift directly.
\begin{equation*} 
\cot((x_1,y_1),(x_2,y_2),(x_3,y_3)) = 
\frac{(x_1-x_2) (x_3-x_2) + (y_1-y_2) (y_3-y_2)}{(y_1-y_2) (x_3-x_2) - (y_3-y_2) (x_1-x_2)}
\end{equation*}


Rewrite \Cref{scaled-dynamics} 
\begin{align}
X_+ - X &= a (x (\cos α - 1) - y \sin α)    &     y_+ - y &= ( x \sin α + y (\cos α - 1))  \\
X_- - X &= a (x (\cos α - 1) + y \sin α)    &     y_- - y &= (-x \sin α + y (\cos α - 1))  
\end{align}
or in trigonometric form
\begin{align}
X_+ - X &= a (\cos(β+α)-\cos(β))    &     y_+ - y &= \sin(β+α)-\sin(β)  \\
X_- - X &= a (\cos(β-α)-\cos(β))    &     y_- - y &= \sin(β-α)-\sin(β)  
\end{align}


and substitute into \Cref{cotangent}
\begin{equation} \label{cot-a}
\cot(x,y,a) = \frac{a x_- x_+ + a^{-1} y_- y_+}{y_- x_+ - y_+ x_-}
\end{equation}
so
\begin{equation}
\cot(x,y,a) = \cot(x,y,1) \cdot \frac{a x_- x_+ + a^{-1} y_- y_+}{x_- x_+ + y_- y_+}
%  \cot(x,y,1) \frac{a (\cos(β+α)-\cos(β))(\cos(β-α)-\cos(β)) + a^{-1} (\sin(β+α)-\sin(β))(\sin(β-α)-\sin(β))}{...}
\end{equation}

Note that 
\begin{align}
x_- x_+ = (x (\cos α - 1) + y \sin α) (x (\cos α - 1) - y \sin α) &= x^2 (\cos α - 1)^2 - y^2 \sin^2 α \\
y_- y_+ = \dots                                                   &= y^2 (\cos α - 1)^2 - x^2 \sin^2 α \\
x_- x_+ + y_- y_+                                                 &= (\cos α - 1)^2 - \sin^2 α 
\end{align}
and since$x=\cosβ$,$y=\sinβ$, 
$x^2 + y^2 = \cos^2 β + \sin^2 β = 1$ (the latter $R=1$ is just a choice for convention)
and $\sin^2 α + \cos^2 α = 1$ we get
\begin{multline}
\int_{β=0}^{2π} \cot(β,a) dβ 
= \frac{\int_{β=0}^{2π} dβ \cdot ( a (A \cos^2 β - B \sin^2 β) + a^{-1} (A \sin^2 β - B \cos^2 β))}{A-B} 
\\
= \left(\int_{β=0}^{2π}\cos^2 β dβ\right) \cdot \frac{aA-a^{-1}B}{A-B}
+ \left(\int_{β=0}^{2π}\sin^2 β dβ\right) \cdot \frac{a^{-1}A-a B}{A-B}
\end{multline}
where $A = (\cos α - 1)^2$ and $B = \sin^2 α$ are constant that don't depend on $β$.

Taking into account
\begin{align*} 
\cos(β)&=\sin(β+π/2) &
\sin^2 β + \cos^2 β &= 1
\end{align*}
one gets 
\[ \int \sin^2 β = \int \cos^2 β = \frac12 \] 
hence
\begin{equation}
\int_{β=0}^{2π} \frac{dβ}{2π}\cdot \frac{\cot(β,a)}{\cot(β,1)}
= \frac12 \cdot \frac{aA-a^{-1}B}{A-B}
+ \frac12 \cdot \frac{a^{-1}A-a B}{A-B}
= \frac{a+a^{-1}}2
\end{equation}

So average over space of $\cot(β,a)$ equals $\frac{a+a^{-1}}{2} \cdot \cot(β,1)$
and $\cot(β,1)$ is easy to compute --- it does not depend on $β$ and equals $-\cot(α)$.

\begin{equation} \label{space-average}
\int_{β=0}^{2π} \frac{dβ}{2π}\cdot \cot(β;a,α) = \frac{a+a^{-1}}{2} \cdot \cot(α)
\end{equation}

Probably my trigonometry is awkward and the whole computation above can be written more nicely.

Now just let us note that almost everywhere instead of

\[    \frac{1}{2π} \int_0^{2π} (dβ)  \cdot \cos(β)^2 \]
we can write
\[    \frac{1}{N}  \sum_{k=0}^{N-1}  \cos\big(\frac{2πk}{N}\big)^2 \]

Using
\[      \cos(β) = \frac{\exp(iβ) + \exp(-iβ)}2 \]
 we have
\[      \cos(β)^2 = \frac12 + \frac14\cdot{(\exp(i\cdot2β) + \exp(-i\cdot2β))} \]

%Em jeito similar, sin(β) = (exp(iβ) - exp(-ib))/(2i) e sin(β)^2 = 1/2 - (exp(i 2β) + exp(-i 2β))/4. [usando sin^2 = 1 - cos^2 esta linha não é necessária]

Thus for $N>2$ we have

\[ \sum_{k=0}^{N-1} \exp\big(2πi\cdot \frac{2k}{N}\big) 
= \frac{1-\exp(2πi\cdot 2k)}{1-\exp(2πi\cdot\frac{2}{N})} 
= \frac0{1-\exp(2πi\cdot\frac{2}{N})} = 0 \]

%[ Se N=1 ou N=2 o denominador é 0, e na verdade a soma não é zero ]

This implies that for $N>2$ periodic time average of $\cos^2(β)$ equals to spatial average,
hence periodic time averages of $\cot(θ)$ are equal to their spatial time averages.

\section{Sine, cosine, and other continuous functions}

Angle $θ(β;α)$ depends continuously on $β,α$,
so $θ$ is a continuous function with values in a circle $S^1 = \R/(2π)$.


Hence the trigonometric functions $\sin(θ)$, $\cos(θ)$ also depend on $β$ continuously.
More generally, for any continuous function $f : S^1 \mapsto \R$
the composition $f(θ)$ is a continuous $\R$-valued function of $β$.

So the ergodic theorem certainly applies to all those cases.
In fact, ergodic theorem does not require continuity, only integrability.
So one can also make some arbitrary cuts, 
like choosing $θ$ to be in range $(-π,π]$, 
but this just looks a little bit unnatural to me.
Or one can consider Fourier expansions.
 $\cos(2θ)$, $\sin(3θ)$, etc --- 
all of them and all their linear combinations will be conserved.
For infinite linear combinations 
\[ \sum_n a(n) \cos(nθ) + b(n) \sin(nθ) \]
some condition on integrability is necessary,
 e.g. $\sum_n |a(n)|^2 + |b(n)|^2 < \infty$ 
should be convergent.


Let $x$ be a point on a boundary conic $D$ 
and $ξ,ξ_2$ be two points on the caustic conic $C$,
such that the lines $\overline{xξ}$, $\overline{xξ_2}$ are tangent to $C$.
Consider $x,ξ,ξ_2$ as complex numbers (by Gauss--Argand identification of Euclidean plane with $\C$).
The function
\[     ν(x;ξ) := \frac{x-ξ_2}{x-ξ} \]
is a $\C$-valued function on
\[ X := \{(x,ξ) | x\in D,\,ξ\in C,\,\overline{xξ}\in T_ξ C\}, \]
the elliptic curve of Poncelet pairs $(x,ξ)$.
With $ξ_2$ being a function of $(x,ξ)$.

A priori this function is only rational --- it has some poles.
However if $X$ is smooth (i.e. conics $C$ and $D$ intersect in $4$ distinct points)
then $ν$ gives a regular map $v: X\to \P^1$ from $X$ to the Riemann sphere $\C\P^1$.

Also in the case when conics $C,D$ are real non-empty and $D$ lies inside the interior of $C$ (like in the pictures)
function $ν(x,ξ)$ clearly takes finite values --- denominator vanishes only when $ξ=x$.
This means $x=ξ$ is one the points of intersection of $C$ and $D$ --- 
these are imaginary 
(by Bezout theorem there are $4$ such points, taking into account multiplicities; 
in smooth case all $4$ are distinct, 
for two concentric circles 
these are two complex-conjugate points 
$(1:\pm\sqrt{-1}:0)$ 
with multiplicity $2$ each).

So restricted to the real locus $X(\R)$, the function $ν$ is in fact regular, continuous and non-vanishing.
Moreover, on the real locus $ν$ does not take any real values --- $ν(x,ξ)$ is real 
iff points $x,ξ,ξ_2$ are collinear, which implies $ξ_2=ξ=x$ being one of the four base-points.
Thus $X(\R)$ has two connected components (left-movers and right-movers,  or counter-clock-wise vs clock-wise),
with values on one component in upper half-plane and on another in lower half-plane of complex numbers.

This corresponds to geometrically obvious fact 
that for the left-movers angle $θ$ is in the range $(0,π)$ 
and for the right-movers $θ$ is in the range $(-π,0)$:
it is contained in the half-plane bounded by $T_x D$ that contains the conics.

With these assumptions one can also unambiguously compute functions as $\frac{θ}2$,  $\sin(\frac{θ}2)$, etc.
And other important observations - $0$ and $π$ are precisely the zeroes for $\sin(θ)$,
so for left-movers $\sin(θ) > 0$ and also $\sin(θ/k) > 0$ for any $k\geq1$, in particular sine are non-zero.
This makes $\cot(θ)$ and also $\cot(θ/2)$, well-defined and continuous. 
Since the circle is compact they are also bounded, integrable, etc.

\section{Questions}

\begin{itemize}
\item Rate of convergence?
\item Experimental verification
\item Falsifiability - for which functions f one cannot apply ET?
\item For example - why cotangents don't work for general conics?
\item Protocol for verification
\end{itemize}


A Non-Example. On hyperbola $xy=c$ the Poncelet map becomes $(x,y) \to (ax,y/a)$.
\begin{enumerate}
\item  For $|a|\neq1$ it acts on the hyperbola properly discontinuously.
\item  The measure is infinite: $\int_{\R} d \log x = \infty$
\item  \[ \frac{\sum_{n=-(N-1)}^{N-1} Cot_n}{2N-1} = Const(a) \cdot \frac{a^{2N}-a^{-2N}}{2N-1}\cdot (x^2+x^{-2}) \]
This goes to infinity when $N\to\infty$ and $|a|\neq 1$
\item Similarly the integral for average over space
$\int_{\R} (x^2+x^{-2}) d \log x$
 per se is divergent.
\end{enumerate}


\begin{example} [THIS IS WRONG]
For (possibly intersecting transversally) two ellipses $C$,$D$ any polynomial of coordinates of $x,ξ$ is conserved.
\end{example}


\section{More remarks}

Poncelet map is discrete, 
but for each particular $C,D$
it is obviously a finite time flow 
for a holomorphic vector field $\frac{\partial}{\partial β}$ on $E$.
Maybe consideration of (the derivation with respect to) 
the respective vector field can give some simplifications.


Speaking about derivations: also one can consider derivation 
with respect to parameter along the moduli space (e.g. along $a$ and $b$).
In cotangent case, the operator
\[  f \to \frac{df}{d(\log a)} = \frac{a df}{(da)} \]
applied twice leaves the quantities (sums and integrals) invariant.


If we consider two different types of derivation,
one along the elliptic curve (analogue of $\frac{\partial}{\partialβ}$)
and another along the moduli space (analogue of $\frac{\partial}{\partial \log a}$)
then there is a nice combination of these two, 
the so-called \emph{heat equation}, 
which I think could we be very useful here.
Essentially it shall be something like 
two derivations along the curve 
is equivalent 
to one derivation along the moduli.


Probably all the relevant functions 
lie in a respective space 
of Jacobi modular forms.
And these spaces are usually finitely generated.
So I think all invariant quantities could be expressed in terms of a few basic ones.


Griffiths and Harris in their two Poncelet-related works \cite{GH1977,GH1978}
cite two papers of Cayley (1853 and 1861).

They write that Jacobi in his solution (1828) 
and Cayley uses slightly too complicated (for GH) 
computations with elliptic functions, 
which they replaced with a purely algebro-geometric proof.

I just found, that Olivier Nash (Hitchin's student) has collected the relevant links and pdfs on his blog post:
\cite{Nash2018,Cayley1853,Cayley1861}

Relation to Painlevé VI. 
In his paper \cite{Hitchin1992},
dedicated to 60 years of Narasimhan and Seshadri,
Hitchin used Poncelet N-gons to construct some algebraic solutions to Painlevé-VI.
I used to know the construction, but forgot them, and just started to re-read the paper.

But I also noticed that recently 
Dragovic and Shramchenko in \cite{DS2019}

generalized this from N-torsion to general elements 
in the elliptic curve to construct 
(non-algebraic) solutions of Schlesinger.
Here are the slides of one of Dragovic talks with brief review:   
\url{https://www.fields.utoronto.ca/programs/scientific/14-15/arnoldconf/slides/dragovic.pdf}


And while composing this mail I noticed a paper \cite{IU2007}.
They cite Hitchin but only in the introduction. 
Looks interesting, but I do not yet see if there is a relation to our story.


Also, Barth--?? writes that 
the Poncelet theorem can be derived 
from a 3d version 
when one of the two quadrics is a cone.
Then one can look for various quantities 
in the 3-space and for their invariance.


\section{Finite fields}

Some experiments with rational functions of homogeneous coordinates of $x,ξ$ for finite fields.
For any pair $C$,$D$ if $X$ is smooth, then $T$ is $N$-periodic for some $N$ automatically ---
over finite field all points on elliptic curve are torsion.
So it makes sense to consider analogous rational functions of coordinates and see whether
their sums are constant (for points where they are defined), and if not - how they vary.

\section{Conic loci of polygonal centers interpreted in terms of elliptic curves}

Dan defines
\begin{quote}
pseudo-circumcenter (system I) and pseudo-orthocenter (system II)
\end{quote}
and experimentally verifies that these are constrained to some circles (conics).

In this section I explain the most plausible origin of all those conics.

{\bf Setting the stage.}

For two given conics $C$ (the so-called caustic)
and $D$ (the so-called boundary),
the incidence space $Χ$ of pairs 
or triples $(P,Q)$ (resp. $(P,L)$ or $(P,Q,L)$) defined by conditions
\begin{enumerate}
\item a point $P$ lies on a boundary conic $D$,
\item a point $Q$ lies on a caustic conic $C$,
\item a point $P$ and a line $L$ are incident, that is a point $P$ lies on a line $L$,
\item a line $L$ is tangent to the caustic conic $C$ in the point $Q$:  $L = T_Q C$
\footnote{So for a fixed non-degenerate caustic $C$,
a point $Q$ is essentially the same thing as a line $L$.
We implicitly identified a plane conic $C\subsetΠ$ with its dual $C^*\inΠ^*$.
In my opinion, all the constructions are easier
(and slightly more general for degenerate cases)
when the two conics are chosen in the two dual planes.
Εven better and more general is to choose two conics in two
a priori unrelated planes
$Π_1 = \P(V_1)$ and $Π_2=\P(V_2)$ and 
then to choose a bilinear form on $V_1\times V_2$
(possibly degenerate).
A non-degenerate form (so-called perfect pairing) gives
an identification of $V_2$ with $V_1^*$,
but generally we have just a complete intersection of three hypersurfaces
of bidegrees $(2,0)$, $(0,2)$ and $(1,1)$ in the Cartesian product $Π_1\times Π_2$
of two projective planes.
Hypersurface of bidegree $(2,0)$ is a cylinder over the conic $D$,
hypersurface of bidegree $(0,2)$ is a cylinder over $C^*$
and hypersurface of bidegree $(1,1)$ is a degeneration of the incidence variety.
}
\end{enumerate}

is a curve $X$ of (arithmetic) genus one equipped with two natural forgetting maps
(or projections)
\begin{align} 
π_C : X\to C  &	&	π_D : X\to D  \\
π_C (P,Q) = Q &	&	π_D (P,Q) = P
\end{align}

Each of the two projections $π_C,π_D$ is a double cover.
These are branched in four points,
the points of intersection of $C$ and $D$,
\footnote{For nested ellipses we don’t see them
because they are two pairs of complex-conjugate points,
but these ''imaginary'' intersection points are there and are important.}
these four points are also known as the base locus.
\footnote{For a pencil (two-dimensional linear system) 
generated by conics $C$ and $D$.
I don't (yet) use pencils in this section.}

\begin{remark} The curve $X$ is smooth if and only if
the conics $C,D$ are smooth \emph{and intersect transversally}, that is
all four their intersection points are distinct.
If some of them are infinitely close (conics are tangent somewhere)
then the curve $X$ is singular.
For example, pair $C,D$ is projectively equivalent (over $\R$)
to a pair of two concentric circles
iff they are tangent to each other in a pair complex conjugate points;
for concentric circles, explicitly,
these are the points $(1:\pm i:0)$ each with multiplicity $2$.
\end{remark}

The covering group of a double cover is generated by an involution,
call these two involutions $ι_C$ and $ι_D$ respectively.
The composition 
\begin{equation} \label{def:tau}
   τ = ι_D \circ ι_C
\end{equation}
of the two involutions is the translation considered by Poncelet.
It is a translation on a genus one curve $Χ$ 
by an element in its Jacobian $Ε := \Jac X = \Pic^0 X$
which I denote by the same letter $τ$.

\begin{remark} Curve $X$ per se is not a group (its real locus could be empty!),
but it is a principal homogeneous space over the group $E$, the curve Jacobian.
This means that there is an action map $E \times X \to X$
such that for any point $x$ on $X$ the respective map $E = E \times \{x\} \to X$
is an isomorphism. So non-canonically $E$ and $X$ are isomorphic,
but only provided $X$ has at least one point.
For example if we choose the boundary $D$ to be an ellipse inside the caustic $C$
(yes, I really mean this - \emph{choose the boundary inside the caustic, not outside}),
then the respective curve $X$ has no real points,
however the Jacobian $E$ always have them.
The whole dynamics is defined over the reals,
but it is slightly hard to visualise dynamics on $X(\R)$ because it is an empty space $\emptyset$.
On the other hand, $E(\R)$ is never empty,
and the dynamics on $E(\R)$
looks almost the same as in the standard case.
\footnote{I imagine it is possible that $E(\R)$ has only one connected component,
but I have not checked this.}
\end{remark}

Poncelet $N$-gons appear exactly when $τ$ is $N$-torsion (an element of order $N$).
Then powers of $τ$ generate a finite cyclic subgroup $C_N = \Z/N$ of $E$.
Natural actions of $E$ on itself and of $E$ on $X$ 
can be restricted to this cyclic subgroup $C_n = \langleτ\rangle \subset E$.
The quotients of $X$ and $E$ with respect to $\langleτ\rangle$ 
are a genus one curve 
\[ X' := X/\langleτ\rangle \]
and its Jacobian 
\[ E' := E/\langleτ\rangle = \Jac X'.\]

{\bf Skip to here if bored by above.}

To a pair of conics \emph{of period $N$}
one associates an elliptic curve $E$
\footnote{More precisely, a genus one curve $X$ and its Jacobian, an elliptic curve $E$, see above.}
equipped with a free action of a finite cyclic group $\langleτ\rangle\simeq C_n \simeq \Z/N$
with generator $τ$.
Freeness implies that the quotient $E' := E/\langleτ\rangle$ is another elliptic curve,
let us denote the respective quotient map as 
\[ ε: Ε\to Ε', \]
this map is also called $n$-isogeny.
Two involutions $ι_C,\,ι_D$ (defined above)
generate a dihedral group $D_n = \langleι_C,ι_D\rangle$
in which $C_n= \langleτ\rangle$ sits as an index two subgroup and 
the quotient-group is generated by either of the two involutions.
\footnote{In planimetric terms $C_n$ is the group of rotations of a regular $n$-gon
and $D_n$ is the group of its isometries ($n$ rotations and $n$ reflections),
this is how $τ$ and $ι$s act in the canonical coordinate.}
Let \[ι: E'\to Ε'\] denote the involution induced by a reflection,
and \[π' : E' \to Ε'/ι =: B\] 
be the respective quotient double cover.
The curve $B$ has genus zero and $π'$ is a double covering branched over four points.
Composition $π'\circε : E \to Β$ is the quotient with respect to the
dihedral group $D_n = \langleι_C,ι_D\rangle$ and $B = E/\langleι_C,ι_D\rangle$.
Note that the curves $C = X/ι_C$ and $D = X/ι_D$
also come naturally equipped with degree $n$ maps to $B$.
Maps $C\to Β$ and $D\to B$ are given by some rational functions of degree $n$
in the respective uniformizers.

The uniformizer on $B$ is the generator of the field of $D_n$-invariant
rational functions on $E$.
A part of the Galois lattice of $D_n: E$ is summarized as
\begin{align}  \label{eq:invariants}
 \k(C) &= \k(E)^{ι_C} \\
 \k(D) &= \k(E)^{ι_D} \\
 \k(E') &= \k(E)^τ    \\
 \k(B) &= \k(E)^{\langleι_C,ι_D\rangle} = \k(E')^ι
\end{align}
here $\k$ is the field of definition of $C,D$.
For the remaining part of the Galois lattice,
note that $ι_C$ and $ι_D = τ \circ ι_C$ are just the first two consecutive reflections
in the set of $n$ reflections 
\[ \{ ι_l := τ^{\circ l} \circ ι_C\,\text{for}\,l=0,1,2,\dots,n-1. \]
We can naturally consider other quotients
\[ C_l := E/ι_l \]
with $C_0 = C$, $C_1 = D$
and the respective invariant fields
\begin{equation}
\k(C_l) = \k(E)^{ι_l}\, l=2,3,\dots,n-1.
\end{equation}
Possibly this means that there is a natural action of a cyclic group $\Z/n$
on the set of Poncelet pairs $(C,D)$, that maps $(C,D)=(C_0,C_1)$ 
to $(C_1,C_2)=(D,C_2)$.
Anyway, curves $C_l$ have genus zero, so each $\k(C_l)$ is generated by
just one uniformizer $x_n$, and for any $n$ we can write down $x_n$ as function
of $x_{n-1},x_{n-2}$.
\begin{remark}The moduli space of Poncelet pairs is the modular curve
$M_1(N)$. 
\footnote{Sometimes it is also known as $X_1(N)$,
but do not confuse it with $X$ that I use throughout this section.}
It naturally has an action of the multiplicative group $(\Z/N)^*$,
an abelian group of order $φ(Ν)$,
but not of the cyclic group of order $N$.
Maybe this means that $(D,C_2)$ as a pair of conics
is projectively equivalent to $(C,D)$,
maybe some explicit collinearity of the plane of order $n$ is involved.
\end{remark}


{\bf Enter Dan maps.}

Let $δ: E\to Z$ be one of the maps from $E$ to one of the “the circle centers” $Z$.
By the very construction of $δ$ we see that $\langleτ\rangle$ is a subgroup of deck transformations
(i.e. $δ\circ τ = δ$).
This means that the map $δ: E\to Z$ factors through $ε: E\to E'$,
that is there exists a unique $δ': E'\to Z$ such that $δ = δ'\circε$.

Moreover, not only $τ$ but each 
individual involution $ι_C$,$ι_D$ (see their definition below)
is in the deck transformation group of $δ$, that is $δ ι_C = δ = δ ι_D$.

This implies that map $δ: E\to Ζ$ factors through $π'\circε: E\to B$,
so there is a map $δ": B\to Z$ between the conics (genus zero curves).
Most likely it is an isomorphism, but apriori could be a map of higher degree,
so a coordinate on $Z$ is a rational function of a coordinate on $B$.


How to check it numerically:
transport the invariant measure from $E$ to the circle center $Z$.


Also, compute the invariants in \eqref{eq:invariants}.

\section{Rational point}

Barry Mazur torion theorem (1978) says that on elliptic curve over rational
$n$-torsion points with rational coefficients exist only for
\[ n=1,2,3,\dots,10,12.\]
Equivalently, it says that the modular
curve $X_1(n)$ has non-cuspidal rational points only for these eleven
values of $n$. This can be equivalently interpreted as a condition
for existence of rational Poncelet $n$-gons.
In fact, exactly for the same values of $n$ the modular curve $X_1(n)$
has genus zero, so rational Poncelet $n$-gons are dense in the moduli
space for this values: for any pair of conics and any Poncelet $n$-gon
associated with them exist a close pair of conics and Poncelet $n$-gon
with every vertex defined over rationals.
\begin{remark} Poncelet $n$-gons for the pair of concentric circles are
regular, so generally Poncelet $n$-gons can be thought
as deformations of regular $n$-gons.
This makes cases $n=7$ and $n=9$ more interesting, since
for these two values usual regular $n$-gons cannot even be constructed
with compass, but on the other hand.
\end{remark}
If not only everything is rational, but also rank of the Mordell--Weil group
of $E$ is more than $0$, then for fixed nice $C$ and $D$ the associated
rational Poncelet $n$-gons are dense in $E(\R)$.

In these cases it is easier to see constancy of invariants:
if $f$ is a rational function of coordinates then it taked rational values
on the rational points.

For quadratic fields this has been generalized by Kenku--Momose (1988)
and Kamienny (1992) to the list of seventeen value
\[ n = 1,\dots,16, 18\]

And the case of torsion over cubic fields wad recently settled
in \arxiv{2007.13929} with twenty one values
\[ n = 1,\dots,20,21. \]
Values of $n\leq 20$ occur for infinitely many moduli of elliptic curves,
while all elliptic curves over cubic fields
with $21$-torsion have the same $j$-invariant
(are isomorphic over $\C$).


\section{Euclidean plane as a projective plane with a pair of points.}
Material of this section is well-known,
and appears in the first courses of geometry at math departments.
I include it just for the completeness.

Main insight - Euclidean planimetry is equivalent to
projective planimetry in presense of fixed pair of points $Z$
on the projective plane $\P^2_\R$. This can be abstracted to other fields.
Also for the field of reals Euclidean (circlic) geometry 
can be deformed to its hyperbolic version through parabolic.
\begin{enumerate}
\item hyperbolic: two points are real and distinct.
In this case functions on $Z$ is a pair of real numbers that are added and multiplied independently:
$Γ(Z) \simeq \R\oplus\R$.
\item parabolic: two points are infinitely close, so we just have
a point and a line passing through it point (point and a tangent direction
on a plane is equivalent to a point and line). In this case functions on $Z$ are dual numbers: $Γ(Z) \simeq \R[ε]/ε^2$.
\item circlic (Euclidean): two points are non-real, but complex conjugate
(so the pair is still real).
In this case functions on $Z$ are complex numbers:
$Γ(Z) \simeq \C$.
\end{enumerate}

\subsection{(Co)tangents as a cross-ratio}
(Co)tangents of the angle between two lines $M$, $N$
essentiallyt equals to the cross-ratio of $4$ points on
the infinite line $L_\infty = (*:*:0)$, 
namely $(1:i:0)$, $(1:-i:0)$, $L\cap M$ and $L\cap N$,
these are two points of the intersection
and the base locus of the linear system of Euclidean circles.

\subsection{Hermite's cotangent identity}

\section{How to construct a moduli space of Euclidean triangles}

Issues.

How to treat degenerate cases (e.g. $Hilb^3 \P^2$ vs other approaches).

Relation to $M_{g=0,n=5}$ and to configurations of $5$ points on a plane.
Cf. Kapranov picture.

Gauss--Argand--B..--Mikhalkin: $z\in\C\backslash\{0,-1\}$,
or for non-degenerate positively oriented case $z\in\H$.
$X+Y+Z=0$, $x=\frac{X}{Z}$, $y=\frac{Y}{X} = \frac{-X-Z}{X}=-1-\frac{1}{x}$,
$z=\frac{Z}{Y} = -1-\frac{1}{y} = \frac{-1}{x+1}$.
Change of cyclic order: $x\to y\to z\to x$.

If a triangle has angles $α_1,α_2,α_3$ with $α_1+α_2+α_3=π$
and $x_i := \cos α_i, y_i := \sin α_i, t_i = \tan α_i, c_i = \cot α_i $.
\begin{gather}
x_1^2 + x_2^2 + x_3 ^2 = 1 - 2 x_1 x_2 x_3 \\
t_1 + t_2 + t_3 = t_1 t_2 t_3 \\
c_2 c_3 + c_1 c_3 + c_1 c_2 = 1
\end{gather}



\printbibliography

\begin{thebibliography}{xxx}
\bibitem{Griffiths-Vienna}
Add link to the popularizing talks of Griffiths.
E.g. his colloquium in Vienna, May 2017.

\bibitem{Cayley-1853}
Arthur Cayley:
\url{http://olivernash.org/2018/07/08/poring-over-poncelet/assets/pdfs/Cayley-1853.pdf}

\bibitem{Cayley-1861}
Arthur Cayley:
\url{http://olivernash.org/2018/07/08/poring-over-poncelet/assets/pdfs/Cayley-1861.pdf}

\bibitem{GH1977}
Philip Griffiths, Joe Harris:
\emph{Poncelet in space}, 
1977 

\bibitem{GH1978}
Philip Griffiths, Joe Harris:
\emph{On Cayley proof of Poncelet porism},
1978 

\bibitem{Hitchin1992}
Nigel Hitchin:
\emph{Poncelet polygons and the Painlevé equations},
\url{https://mathscinet.ams.org/mathscinet-getitem?mr=1351506} 
\url{http://olivernash.org/2018/07/08/poring-over-poncelet/assets/pdfs/Hitchin-1995.pdf}

\bibitem{IU2007}
Iwasaki--Uehara
\emph{An ergodic study of Painlevé VI} 
\url{https://mathscinet.ams.org/mathscinet-getitem?mr=2302065} 
\url{https://doi.org/10.1007/s00208-006-0077-8}

\bibitem{Nash2018}
Oliver Nash:
\url{http://olivernash.org/2018/07/08/poring-over-poncelet/index.html}

\bibitem{DS2019}
Dragovic and Vasilisa Shramchenko:
\url{https://mathscinet.ams.org/mathscinet-getitem?mr=3928374}
\url{https://doi.org/10.1007/s00023-019-00765-8}

\end{thebibliography}


\end{document}
