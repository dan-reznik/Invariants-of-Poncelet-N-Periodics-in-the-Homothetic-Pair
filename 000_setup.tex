\usepackage{graphics}
\usepackage{amsthm}
\usepackage{thmtools}
%\usepackage{wasysym}
%\usepackage{arxiv}
%\usepackage{enseign}

\usepackage{iftex}                      % check which engine is compiling this file
\iftutex                                % xelatex or lualatex specific first
 \usepackage{fontspec}                  % font selection in xelatex/lualatex
 \defaultfontfeatures {Ligatures=TeX}
 \usepackage{unicode-math}              % amssymb+amsmath equiv., to show $θ$ as θ
 \unimathsetup{math-style=TeX}
 \setmathfont{Stix Two Math}
 \ifLuaTeX
  % \usepackage{luatex85}               % compatibility of luatex with xy
   \usepackage{lualatex-math}
 \fi
\else                                   % 8bit (e.g. pdflatex) specific
 \usepackage[T1]{fontenc}               % use 8-bit T1 fonts
 % \usepackage[utf8]{inputenc}           % allow utf-8 input, for texlive-2016 and older
 \usepackage{amsmath,amssymb}
\fi
\usepackage{alphabeta}                  % type αβγ directly, in LuaLaTeX use {unicode-math} (above)


\usepackage{amsbsy,amscd,amsfonts}
\usepackage[pagebackref=true]{hyperref}
\usepackage[nameinlink,capitalize,noabbrev]{cleveref}
%\usepackage{hyperref}
%\usepackage{url}            % simple URL typesetting
%\usepackage{booktabs}       % professional-quality tables
%\usepackage{nicefrac}       % compact symbols for 1/2, etc.
%\usepackage{microtype}      % microtypography

\usepackage{graphicx,float,latexsym,color}
%\usepackage{refcheck}
%\usepackage{mathspec}
\usepackage[font={scriptsize,it}]{caption}
\usepackage{subcaption}
%\let\proof\relax
%\let\endproof\relax

\usepackage{makecell}

% usar \defcommand em vez de \newcommand ou \renewcommand
% funciona como \def em tex - vai definir, se ainda não definida,
% se já definida, vai redefinir
\makeatletter\def\defcommand{\@ifstar\defcommand@S\defcommand@N} \def\defcommand@S#1{\let#1\outer\renewcommand*#1} \def\defcommand@N#1{\let#1\outer\renewcommand#1} \makeatother

\renewcommand{\arraystretch}{1.2}

\usepackage[dvipsnames]{xcolor}

\newtheorem{theorem}{Theorem}
\newtheorem*{theorem*}{Theorem}
\newtheorem{lemma}{Lemma}
\newtheorem{proposition}{Proposition}
\newtheorem{corollary}{Corollary}

\theoremstyle{remark}
\newtheorem{remark}{Remark}
\newtheorem{observation}{Observation}
\newtheorem{example}{Example}

\theoremstyle{definition}
\newtheorem{definition}{Definition}
\newtheorem{question}{Question}
\newtheorem{conjecture}{Conjecture}

\hypersetup{
    %bookmarks=true,         % show bookmarks bar?
    %unicode=true,          % non-Latin characters in Acrobat’s bookmarks
    pdftoolbar=true,        % show Acrobat’s toolbar?
    pdfmenubar=true,        % show Acrobat’s menu?
    pdffitwindow=false,     % window fit to page when opened
    pdfstartview={FitH},    % fits the width of 
    colorlinks=true,        % false: boxed links; true: colored links
    linkcolor=OliveGreen,   % color of internal links (change box color with linkbordercolor)
    citecolor=blue,         % color of links to bibliography
    filecolor=black,        % color of file links
    urlcolor=TealBlue       % color of external links
}

\usepackage{lineno}
\def\linenumberfont{\normalfont\small\sffamily}
%a4: 210 x 297
%\textwidth=125mm
%\textheight=195mm
\arraycolsep=2pt
\captionsetup{width=120mm}

 
\usepackage{comment}
\usepackage{microtype}
\usepackage{footnote}
%\usepackage{tablefootnote}
%\usepackage{longtable}

\usepackage[mark]{gitinfo2}

% my local settings for night reading
% \usepackage{sergey}

%% math blackboard
% \newcommand{\D}{\mathbb{D}}           % unused

%% math bold face
\defcommand{\k}{\mathbf{k}}             % base field
\defcommand{\A}{\mathbf{A}}             % [A]ffine plane/space
% \defcommand{\F}{\mathbf{F}}             % [F]inite [f]ield       (unused)
\defcommand{\C}{\mathbf{C}}             % [C]omplex numbers
% \defcommand{\P}{\mathbf{P}}		% projective plane/space   (unused)
\newcommand{\Q}{\mathbf{Q}}             % Rational numbers
\newcommand{\R}{\mathbf{R}}             % [R]eal numbers
\newcommand{\Z}{\mathbf{Z}}             % Integer [Z]ahlen

%% math calligraphic
\defcommand{\AA}{\mathcal{A}}           % For \AA_s in evolute
\newcommand{\EE}{\mathcal{E}}		% [E]llipse
\newcommand{\FF}{\mathcal{F}(a,b)}           % [F]amily
\newcommand{\PP}{\mathcal{P}}           % [P]olytope
\newcommand{\QQ}{\mathcal{Q}}           % Regular polytope (maybe better R?)
% \newcommand{\TT}{\mathcal{T}}		% unused
% \newcommand{\XX}{\mathcal{X}}         % unused

\newcommand{\ol}{\overline}

\newcommand{\torp}[2]{\texorpdfstring{#1}{#2}}
% \newcommand{\ab}{_{\alpha,\beta}}     % unused

\DeclareMathOperator{\ord}{ord}
\DeclareMathOperator{\Ker}{Ker}
\DeclareMathOperator{\tr}{Tr}
\DeclareMathOperator{\GL}{GL}           % [G]eneral [L]inear group
\DeclareMathOperator{\SO}{SO}           % [S]pecial [O]rthogonal group
\DeclareMathOperator{\SL}{SL}           % [S]pecial [L]inear group
% \DeclareMathOperator{\O}{O}             %           [O]rthogonal group
\DeclareMathOperator{\Un}{U}             %           [U]nitary group

\newcommand{\Mod}[1]{\ (\mathrm{mod}\ #1)}

\newcommand{\complain}[1]{\textcolor{red}{#1}}
