\usepackage{graphics}
\usepackage{thmtools}
\usepackage{wasysym}
%\usepackage{arxiv}
%\usepackage{enseign}

\usepackage{iftex}                      % check which engine is compiling this file
\iftutex                                % xelatex or lualatex specific first
 % \usepackage{fontspec}                % font selection in xelatex/lualatex
 \defaultfontfeatures {Ligatures=TeX}
 \usepackage{unicode-math}              % amssymb,amsmath eq, to show $θ$ as θ
 \unimathsetup{math-style=TeX}
 \setmathfont{Stix Two Math}
 \ifLuaTeX
  %  \usepackage{luatex85}              % compatibility of luatex with xy
   \usepackage{lualatex-math}
 \fi
\else                                   % 8bit (e.g. pdflatex) specific
\usepackage[T1]{fontenc}                % use 8-bit T1 fonts
%\usepackage[utf8]{inputenc}       % allow utf-8 input, for texlive-2016 and older
\fi
\usepackage{alphabeta}              % in LuaLaTeX use {unicode-math}

\usepackage{amsthm}
\usepackage{amsbsy,amsmath,amssymb,amscd,amsfonts}
\usepackage[pagebackref=true]{hyperref}
\usepackage[nameinlink,capitalize,noabbrev]{cleveref}
%\usepackage{hyperref}
%\usepackage{url}            % simple URL typesetting
%\usepackage{booktabs}       % professional-quality tables
%\usepackage{nicefrac}       % compact symbols for 1/2, etc.
%\usepackage{microtype}      % microtypography

\usepackage{graphicx,float,latexsym,color}
%\usepackage{refcheck}
%\usepackage{mathspec}
\usepackage[font={scriptsize,it}]{caption}
\usepackage{subcaption}
%\let\proof\relax
%\let\endproof\relax

\usepackage{makecell}

% usar \defcommand em vez de \newcommand ou \renewcommand
% funciona como \def em tex - vai definir, se ainda não definida,
% se já definida, vai redefinir
\makeatletter\def\defcommand{\@ifstar\defcommand@S\defcommand@N} \def\defcommand@S#1{\let#1\outer\renewcommand*#1} \def\defcommand@N#1{\let#1\outer\renewcommand#1} \makeatother

\renewcommand{\arraystretch}{1.2}

\usepackage[dvipsnames]{xcolor}

\newtheorem{theorem}{Theorem}
\newtheorem*{theorem*}{Theorem}
\newtheorem{observation}{Observation}
\newtheorem{proposition}{Proposition}
\newtheorem{conjecture}{Conjecture}
\newtheorem{corollary}{Corollary}
\newtheorem{lemma}{Lemma}
\theoremstyle{remark}
\newtheorem{remark}{Remark}
\theoremstyle{definition}
\newtheorem{definition}{Definition}
\newtheorem{question}{Question}
\hypersetup{
    %bookmarks=true,         % show bookmarks bar?
    %unicode=true,          % non-Latin characters in Acrobat’s bookmarks
    pdftoolbar=true,        % show Acrobat’s toolbar?
    pdfmenubar=true,        % show Acrobat’s menu?
    pdffitwindow=false,     % window fit to page when opened
    pdfstartview={FitH},    % fits the width of 
    colorlinks=true,       % false: boxed links; true: colored links
    linkcolor=OliveGreen,          % color of internal links (change box color with linkbordercolor)
    citecolor=blue,        % color of links to bibliography
    filecolor=black,      % color of file links
    urlcolor=red           % color of external links
}

\usepackage{lineno}
\def\linenumberfont{\normalfont\small\sffamily}
%a4: 210 x 297
%\textwidth=125mm
%\textheight=195mm
\arraycolsep=2pt
\captionsetup{width=120mm}

 \newcommand{\Mod}[1]{\ (\mathrm{mod}\ #1)}
 
\usepackage{comment}
\usepackage{microtype}
\usepackage{footnote}
%\usepackage{tablefootnote}
%\usepackage{longtable}

%\usepackage{night}

\newcommand{\E}{\mathcal{E}}
\newcommand{\T}{\mathcal{T}}
%\newcommand{\C}{\mathcal{C}}
\defcommand{\C}{\mathcal{C}}
\newcommand{\F}{\mathcal{F}}
\renewcommand{\P}{\mathcal{P}}
\newcommand{\D}{\mathbb{D}}
\renewcommand{\T}{\mathbb{T}}
\newcommand{\R}{\mathbb{R}}
\newcommand{\Cp}{\mathbb{C}}
\newcommand{\ol}{\overline}
\renewcommand{\l}{\lambda}
\newcommand{\X}{\mathcal{X}}

\newcommand{\torp}[2]{\texorpdfstring{#1}{#2}}
\newcommand{\ab}{_{\alpha,\beta}}
