
Let $i$ denote imaginary root of unity, so that $i^2=-1$.
Since de Moivre a rotation of a unit circle 
\begin{equation}
S^1 = \{(r,\varphi):r=1\} = \{(x,y): x^2+y^2=1\} = \{z: |z|=1\}
\end{equation}
can be parameterized
either (i) by its angle $\varphi$,
   or (ii) by a pair of Cartesian co-ordinates $(x,y)$ such that $x^2+y^2=1$,
  or (iii) by a unit complex number $z$,
and the parameterizations are related to each other
and to trigonometric functions $\cos,\sin$ by formulae: 
\begin{gather}
\label{eq:demoivre}
z = x + y \cdot i \qquad = \exp(i \varphi) = \cos(\varphi) + \sin(\varphi) \cdot i, \\
\label{eq:demoivre2}
\cos(\varphi) = (z+z^{-1})/2 = x, \qquad \sin(\varphi) = (z-z^{-1})/(2i) = y.
\end{gather}
Fixing a reference point $P_0$ one can identify a rotation $\rho$
with a point $\rho(P_0)$.
Thus the composition of rotations as well as the action of a rotation on a point
are described by multiplication of complex numbers.
In particular the rotation $ρ$ by a fixed angle $α = \arg ζ$ is a linear transformation
either in complex co-ordinate 
\begin{equation}
\label{eq:dynamics-complex}
ρ : z\mapsto ζ\cdot z
\end{equation}
or in the real co-ordinates $x,y$
(note that $\bar{ρ(z)} = \bar{ζ}\cdot\bar{z}$ is linear in $\bar{z}$ and $z^{-1} = \bar{z}$ on $S^1$,
so the transformations \cref{eq:demoivre,eq:demoivre2} restricted to $S^1$ are linear).
Also, the sequence of points $\dots,P_{k-1},P_k,P_{k+1},\dots$
is obtained by consecutive fixed rotations $ρ$ (i.e. $P_{k+1} = ρ P_k$ for some $ρ$)
if and only if the corresponding sequence of complex numbers
is a geometric progression:
\[ P_k^2 = P_{k-1} \cdot P_{k+1}. \]

Define a scalar product of two complex-valued functions on the circle by
$(f,g) := \int_{S^1} f \cdot \ol{g} \, dμ_L$, where
$dμ_L$ is the Lebesgue measure
$dμ_L = \frac{d\varphi}{2\pi} = \frac{1}{2πi} \frac{dz}{z}$. 
Laurent monomials form an orthonormal basis
in the Hilbert space $L^2(S^1,\C)$ of functions $f$ with $(f,f)<\infty$,
i.e. any such function has a unique Fourier series
\[ f(z) = \sum_{m\in\Z} \widehat{f}(m) z^m, \]
with Fourier coefficients $\widehat{f}(m) = (f,z^m) = \int_{S^1} f(z) z^{-m} dμ_L$.
A Slightly less convenient orthonormal basis for $L^2(S^1)$ is given by the functions
$1,x_1,y_1,x_2,y_2,\dots$ where $x_n := \Re{z^n} = \frac{z^n+z^{-n}}2 = \cos(n\varphi)$
and $y_n := \Im{z^n} = \frac{z^n - z^{-n}}{2i} = \sin(n\varphi)$.
Its main advantage is that $x_n,y_n$ take real values on $S^1$,
so they form a real basis for the real Hilbert space $L^2(S^1,\R)$. In terms of Fourier coefficients
this real structure is given by $\widehat{f}(-m) = \overline{\widehat{f}(m)}$.
Finally, one can also use a (non-orthogonal) basis $x^n,x^n\cdot y$ ($n\geq 0$),
and the transformation to the basis $x_n,y_n$ is given
by the unique degree-$n$ polynomials $T_n$ (Chebyshëv) and $U_n$ (Kórkin--Zolotarëv)
such that $x_n = T_n(x)$ and $y_{n+1} = y \cdot U_n(x)$.

The reflection $ι$ with respect to the real line is given on a unit circle by
$\varphi\mapsto -\varphi$ or $(x,y)\mapsto (x,-y)$ or $z\mapsto z^{-1} = \bar{z}$.
Any function $f$ on $S^1$ is decomposed into the sum $f = f_+ + f_-$
of a $ι$-invariant (``even'') part $f_+(z) = \frac{f(z)+f(z^-1)}{2}$
and a $ι$-anti-invariant (``odd'') part $f_+(z) = \frac{f(z)-f(z^-1)}{2}$.
The even part is generated by the Joukovsky function $J(z) := x = x_1 = \frac{z+z^{-1}}{2}$
and is naturally isomorphic to the functions on the image: 
(either the unit interval $[-1,1]$, or the affine/projective line, depending on the context).
This is a subalgebra (sum and product of invariant functions is invariant),
and in terms of Fourier coefficients it is given by $\widehat{f}(-m) = \widehat{f}(m)$.
The odd part is a free cyclic module over this algebra
with a generator $y = y_1$,
i.e. any odd function equals to $y$ times a unique even function.

For any bivariate polynomial of degree less than $d$,
the result of the substitution $(x,y) \mapsto (\cos(\varphi),\sin(\varphi))$
can be uniquely written as a linear combination of Laurent monomials $z^k$ 
or $x_k,y_k$ with $|k|<d$,
We will call such elements trigonometric (or Laurent) polynomials of degree less than $d$.
We will further label them ``even'' or ``odd'' if they are invariant or anti-invariant
with respect to the involution $ι$.
An example of an odd rational function which is 
not square-integrable (in particular, not a trigonometric polynomial)
is given by the cotangent:
\begin{equation}
\small
 \label{eq:cotangent}
\cot(\varphi) := \frac{\cos(\varphi)}{\sin(\varphi)} 
= \sum_{n\in\Z} \frac1{\varphi+n\cdotπ}
= i \left(1+\frac{1}{z-1}-\frac{1}{z+1}\right)
% = i \left(1+\frac{2}{z^2-1}  \right) 
= i \frac{z^2+1}{z^2-1} 
=: \xi(z) 
\end{equation}

