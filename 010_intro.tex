% 

Let $\FF$ be a family of affine images of a regular polygon
whose circumcircle is sent to an ellipse with fixed semi-axes of length $a,b$.

Let us say that a \emph{trace} of a ''quantity'' is the arithmetic mean of its values on the vertices of the polygon in $\FF$.
A priori the trace depends on the chosen polygon and/or on the particular transformation $T\in\FF$.
We will explicitly compute the traces of the ''interesting'' quantities below.
In particular, we prove that these are constant over $\FF$
and compute their limits for infinite gonality (denoted further by $N$):

\begin{itemize} \itemsep0em
    \item \eqref{:norm2}: the squared distance to any fixed point; see \cref{fig:homot_norm}.
    \item \eqref{:sqr_si}: the squared sidelength.
    \item \eqref{:cot}: the cotangent of an internal angle.
    \item \eqref{:dist-focus}: the distance to a focus of $\EE$.
    \item \eqref{:kappa}: \textcolor{magenta}{the kinetic energy} $\kappa^{-2/3}$, where $\kappa$ is the curvature of the ellipse $\EE$.
    \item \eqref{:ell2}: the squared sidelength of the evolute polygon (except squares and hexagons). 
    \item \eqref{:as}: the signed area of the evolute polygon (except squares).
\footnote{We explain later in which sense area can be thought as a function.}
\end{itemize}

As a part of the computation
we verify that each of the traced quantities is a Laurent polynomial
when expressed in terms of the natural coordinate on the circle.
Then the application of \cref{:Laurent}
implies that its trace is constant for all gonality higher than the degree of the polynomial,
and the remaining traces are explicitly computed in terms of coefficients by \cref{:fourier-of-trace}. The degree for each Laurent polynomial appears on \cref{tab:invariants-degrees}.

\subsection*{Article organization}
We start from \cref{sec:polygonomials}
by defining the polygons and (functions on) their moduli spaces,
and explaining our selection criteria
for the ''quantity'' to be ''interesting''.
After this section it will be clear why the kinetic energy
is considered to be an interesting quantity,
but coordinates and momenta (whose traces are also constant) are not:
the former is invariant with respect to the Euclidean transformation group,
while the latter are covariant (equivariant).

In \cref{sec:trigo}
we give trigonometric preliminaries,
then introduce the traces
in \cref{sec:traces},
and characterize functions with constant trace
by \cref{:fourier-of-trace,:functoriality,:Laurent,:characterization,:strange,:ergodic},
from which all the remaining claims follow by straightforward computations,
which are described in \cref{sec:examples}.
In \eqref{sec:areas} we discuss oriented areas,
many natural oriented areas are constant even before the tracing,
and also we explain how they split into local summands.
In \eqref{sec:scalar} we discuss functions that are quadratic polynomials for obvious reasons --- scalar products of linear transformations.
In \eqref{sec:cot} we discuss cotangents that also turn out to be quadratic.
In \eqref{sec:focuspocus} we show that distance to the focus is linear
and $κ^{-2/3}$ is quadratic: a priori we knew only that their square and cube are polynomials,
but these polynomials turned out to be
a perfect square (of a coordinate)
and a perfect cube (of the kinetic energy).
In \eqref{sec:evolute} we compute the quantities
associated with evolute polygons (of degrees four and six)
using an explicit parametrization
for the evolute\footnote{The curve sweeped by centers of curvature (centers of the oscullating circles).},
of the ellipse, a curve affinely equivalent to the \href{https://en.wikipedia.org/wiki/Astroid}{astroid}
\footnote{Plane algebraic curve $x^{2/3}+y^{2/3}=1$,
see e.g. \url{https://en.wikipedia.org/wiki/Astroid},
\url{https://mathshistory.st-andrews.ac.uk/Curves/Astroid},
\url{http://xahlee.info/SpecialPlaneCurves_dir/Astroid_dir/astroid.html}}.

In \cref{sec:video-table} we (i) 
list the polynomial quantities sorted by their degrees
and hypothesize on what could be their ``analogues'' in the elliptic billiard
and (ii) provide a list of illustrative videos of the phenomena herein.
For convenience, all used symbols appear in \cref{app:symbols}.



