Let $a_i$ and $b_i$ the axes of the elliptic locus $X_i$, $i=1{\ldots}200$.

In \cite{garcia2020-ellipses} we checked that within $X_1$ to $X_{100}$ 29 centers yield elliptic loci, to be sure: $X_i$, $i=$1, 2, 3, 4, 5, 7, 8, 10, 11, 12, 20, 21, 35, 36, 40, 46, 55, 56, 57, 63, 65, 72, 78, 79, 80, 84, 88, 90, 100. We also provide explicit expressions for their axes.

Extending the list till $X_{200}$, also elliptic are the loci of $X_j$, $j=$104, 119, 140, 142, 144, 145, 149, 153, 162, 165, 190, 191, 200.

\subsection{System I: Incircle}

%I. Incircle Ellipses: {2,3,4,5,7,8,9,10,11,12,20,21,30,35,36,40,46,55,56,57,63,65,72,78,79,80,84,90,100,104,119,140,142,144,145,149,153,165,170,176,191,200}
%       I. Circles: {3,5,11,12,35,36,40,46,55,56,57,65,80,119,165}

\subsection{System II: Inellipse}

%II. Inellipse Ellipses: {2,4,5,6,15,20,21,22,23,24,25,26,27,28,39,49,51,52,54,58,61,64,66,67,68,69,70,74,98,99,100,101,103,104,105,106,107,108,109,110,111,112,113,125,140,141,143,146,154,155,156,159,161,182,184,185,186,193,195,199}
%       II. Circles: {2,4,5,20,22,23,24,25,26,74,98,99,100,101,103,104,105,106,107,108,109,110,111,112,140,156,186,201}
% IV. Dual Ellipses: {2,3,5,20,64,107,122,133,140,154}

\subsection{System III: Homothetic}
\label{app:ell-axes-III}

% III. Homothetic Ellipses: {3,4,5,6,13,14,15,16,17,18,20,32,39,61,62,69,76,83,98,99,114,115,140,141,147,148,182,187,190,193,194}
% III. Circles: {13,14,15,16}

\begin{align*}
	a_3=& \frac{a^2-b^2}{4a}, b_3=\frac{a^2-b^2}{4b}\\
	a_4=&\frac{a^2-b^2}{2a}, b_4=\frac{a^2-b^2}{2b}\\
	a_5=& \frac{a^2-b^2}{8a}, b_5=\frac{a^2-b^2}{8b} \\
	a_6=&\frac{a(a^2-b^2)}{2(a^2+b^2)}  ,b_6=\frac{b(a^2-b^2)}{2(a^2+b^2)} \\
	a_{13}=& \frac{a-b}{2}, b_{13}= \frac{a-b}{2}\\
	a_{14}=& \frac{a+b}{2}, b_{14}= \frac{a+b}{2}\\
	a_{15}=& \frac{(a-b)^2}{2(a+b)} , b_{15}= \frac{(a-b)^2}{2(a+b)}\\
	a_{16}=&\frac{(a+b)^2}{2(a-b)},     b_{16}=\frac{(a+b)^2}{2(a-b)}\\
	a_{17}=&\frac{a^2-b^2}{2a+6b)},     b_{17}= \frac{a^2-b^2}{6a+2b)}\\
	a_{18}=&\frac{a^2-b^2}{2(a-3b)},     b_{18}= \frac{a^2-b^2}{2(3a-b)}\\
	a_{20}=&\frac{a^2-b^2}{a},     b_{20}=\frac{a^2-b^2}{b}\\
	a_{32}=& \frac {a \left( a^2-b^2 \right)   \left( 3\,{a}^
			{2}+5\,{b}^{2} \right) }{2(3\,{a}^{4}+2\,{a}^{2}{b}^{2}+3\,{b}^{4})},  b_{32}= \frac {b \left( a^2-b^2 \right)   \left( 3\,{a}^
			{2}+5\,{b}^{2} \right) }{2(3\,{a}^{4}+2\,{a}^{2}{b}^{2}+3\,{b}^{4})}\\
	a_{39}=&\frac { \left( {a}^{2}-{b}^{2} \right) a}{2( {a}^{2}+3\,{b}^{2})},
	b_{39}=\frac{ \left( {a}^{2}-{b}^{2} \right) b}{2(3 \,{a}^{2}+\,{b}^{2})}\\
	a_{61}=& \frac {    \left( a^2-b^2\right)  \left( 3\,a-b
			\right) }{2( 3a^{2}+2\,ab+3\,{b}^{2})},
		%
		b_{61}= \frac { \left( a^2-b^2 \right)  \left( a-3\,b \right)    }{2(3\,{a}^{2}+2\,ab+3\,{b}^{2})}\\
		a_{62}=& ,{\frac {  \left( a^2-b^2 \right)  \left( 3\,a+b
				\right) }{2(3\,{a}^{2}-2\,ab+3\,{b}^{2})}},
			b_{62}= ,{\frac {  \left( a^2-b^2 \right)  \left( a+3b
					\right) }{2(3\,{a}^{2}-2\,ab+3\,{b}^{2})}}\\
				a_{69}=&{\frac { \left( a^2-b^2 \right)    a}{{a}^{2}+{b}^{2}}},
				b_{69}=\frac { \left( a^2-b^2 \right)    b}{{a}^{2}+{b}^{2}}\\
		a_{76}=&  \frac { \left( a^2-b^2 \right)    a}{{a}^{2}+3{b}^{2}},  b_{76}= \frac{ \left( a^2-b^2 \right)    b}{3{a}^{2}+{b}^{2}}\\
		a_{83}=&  \frac { \left( a^2-b^2 \right)    a}{5{a}^{2}+3{b}^{2}},b_{83}= \frac{ \left( a^2-b^2 \right)    b}{3{a}^{2}+5{b}^{2}}\\
		a_{98}=& \frac{a^2+b^2}{2a}, 	b_{98}= \frac{a^2+b^2}{2b}\\
		a_{99}=&a,\;\;
		b_{99}=b
\end{align*}

\begin{align*}
a_{114}=& \frac{a^2+b^2}{4a}, 	b_{114}= \frac{a^2+b^2}{4b}\\
a_{115}=& \frac{a }{2}, 	\;\; b_{115}= \frac{b}{2}\\
				%
a_{140}=& \frac{a^2-b^2}{16a}, 	b_{140}= \frac{a^2-b^2}{16b}\\
a_{141}=&\frac{(a^2-b^2)a}{4(a^2+b^2)},
b_{141}= \frac{(a^2-b^2)b}{4(a^2+b^2)}\\
a_{147}=&\frac{a^2+b^2}{a},
					b_{147}= \frac{a^2+b^2}{b}\\
					a_{148}=&2{a},
					b_{148}=2b\\
		a_{182}=&\frac{(a^2-b^2)^2}{8a(a^2+b^2)},\;\;
					b_{182}= \frac{(a^2-b^2)^2}{8b(a^2+b^2)}\\
					a_{187}=& \frac{a(a^2+3b^2)}{2(a^2-b^2)},
						b_{187}=  \frac{b(3a^2+b^2)}{2(a^2-b^2)}\\
						a_{190}=& a,\;\; b_{190}=b\\
		a_{193}=&\frac{2(a^2-b^2)a}{a^2+b^2},\;
		b_{193}=\frac{2(a^2-b^2)b}{a^2+b^2}\\
		a_{194}=&\frac{2(a^2-b^2)a}{a^2+3b^2},
		b_{194}=\frac{2(a^2-b^2)b}{3a^2+b^2}				
\end{align*}

\subsection{System IV: Dual}

% IV. Dual Ellipses: {2,3,5,20,64,107,122,133,140,154}