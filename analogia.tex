\section{Analogia}

\textbf{A problem.} The claim that quantities on RHS of \ref{tab:invariants-degrees}
are analogues of the quantities on LHS
is (so far) \emph{completely unfounded}.

\emph{Possible solution.}
There is a well defined mathematical procedure
of limiting (also known as specialization) that can make sense of this.
For this just need to compute limit/specialization
of the RHS for confocal pair limiting to the cusp.


\textcolor{OliveGreen}{Dan: I understand, so should we remove it? They look pleasantly similar that's why I wanted to include them.}


''Pleasantly similar'' is an esthetic (and subjective) category, which is neither scientific nor mathematical.
                      
Rather then removing it,
one shall do an experiment (computation) and see:
what are the limits at the cusps of the quantities on the RHS?

One obvious problem are gradings: they are inverse for $d(P,f_\pm)^\pm$ and $κ^{\mp 2/3}$,
so probably it is better to renormalize by dividing by $(ab)^{grading}$,
then the respective quantities are scale-invariant on both sides,
so just compute the limit oh rescaled RHS and compare with rescaled LHS. 


\textcolor{OliveGreen}{Dan: I like it, but we will need you help with the computation}

