\subsection{Theorem 1}

Below we compute constant $T_N(\cot\theta)$. We use the following parametrization for the vertices.
Rotation by angle $\alpha$ is given by
\[ z\to z' = e^{i \alpha} \cdot z \]
Since $e^{i \alpha} = \cos \alpha + i \sin \alpha$
for $z = x + i y$ and $z'=x'+iy'$ we have
\[
   x'+iy' = z' 
 = e^{i \alpha} \cdot z
 = (\cos \alpha + i \sin \alpha) (x + i y) 
 = (x \cos \alpha - y \sin \alpha) + i (x \sin \alpha + y \cos \alpha)
\] 
or
\begin{align*} x' &= (x \cos \alpha - y \sin \alpha)    &     y' &= (x \sin \alpha + y \cos \alpha) \end{align*}
Also if $z = e^{i \beta}$ then $z' = e^{i (\beta+\alpha)}$ and respectively $x'=\cos(\beta+\alpha), y'=\sin(\beta+\alpha)$.
So we can explicitly write down coordinates for the next and the previous points using $\cos(-\alpha)=\cos(\alpha)$ and $\sin(-\alpha)=-\sin(\alpha)$.
If $T (x_i,y_i) = (x_{i+1},y_{i+1})$, then
\begin{align*} 
x_{i+1} &= (x_i \cos \alpha - y_i \sin \alpha)    &     y_{i+1} &= (x_i \sin \alpha + y_i \cos \alpha)  \\
x_{i-1} &= (x_i \cos \alpha + y_i \sin \alpha)    &     y_{i-1} &= (-x_i \sin \alpha + y_i \cos \alpha)  
\end{align*}

And if $X_i = a x_i$ then

\begin{align} 
X_{i+1} &= a(x_i \cos \alpha - y_i \sin \alpha) & y_{i+1} &= b( x_i \sin \alpha + y_i \cos \alpha) \nonumber \\
X_i &= a x_i & y_i &= b y_i \label{eq:scaled-dynamics} \\
X_{i-1} &= a(x_i \cos \alpha + y_i \sin \alpha)   &    y_{i-1} &= b(-x_i \sin \alpha + y_i \cos \alpha) \nonumber
\end{align}

The cotangent of an angle $\theta=\angle{ABC}$ is rational on the vertex coordinates.
Explicitly, if $B=O:=(0,0)$ and $C=H:=(*,0)$, $A=(x,y)$ then
\[ \cot\theta = \frac{x}{y} \]
If $C=(x_1,y_1)$ and $A=(x_3,y_3)$ consider the associated complex numbers $x_1 + i y_1$ and $x_3 + i y_3$, respectively. Their ratio $d$ is given by:
\[ d 
 = \frac{x_1 + i y_1}{x_3 + i y_3} 
 = \frac{(x_1 + i y_1)(x_3 - i y_3)}{(x_3 + i y_3)(x_3 - i y_3)} 
 = \frac{(x_1 x_3 + y_1 y_3) + i (y_1 x_3 - y_3 x_1)}{x_3^2 + y_3^2}
\]
Angle $AOC$ is homothetic to angle $HOD$, hence:

\begin{equation} \label{cotangent}
\cot\theta = \frac{x_1 x_3 + y_1 y_3}{y_1 x_3 - y_3 x_1} 
\end{equation}

The formula for the cotangent of an abstract triple is obtained by a shift,
but it will be easier to do the shift directly.

\begin{equation*} 
\cot((x_1,y_1),(x_2,y_2),(x_3,y_3)) = 
\frac{(x_1-x_2) (x_3-x_2) + (y_1-y_2) (y_3-y_2)}{(y_1-y_2) (x_3-x_2) - (y_3-y_2) (x_1-x_2)}
\end{equation*}

Without loss of generality we will set $b=1$. Rewrite \cref{eq:scaled-dynamics} as: 

\begin{align*}
X_+ - X &= a (x (\cos \alpha - 1) - y \sin \alpha)& y_+ - y &= ( x \sin \alpha + y (\cos \alpha - 1))  \\
X_- - X &= a (x (\cos \alpha - 1) + y \sin \alpha) & y_- - y &= (-x \sin \alpha + y (\cos \alpha - 1))  
\end{align*}

\noindent or in trigonometric form:

\begin{align*}
X_+ - X &= a (\cos(\beta+\alpha)-\cos(\beta))    &     y_+ - y &= \sin(\beta+\alpha)-\sin(\beta)  \\
X_- - X &= a (\cos(\beta-\alpha)-\cos(\beta))    &     y_- - y &= \sin(\beta-\alpha)-\sin(\beta)  
\end{align*}

\noindent substitute the above into 
\cref{cotangent} and get:

\begin{equation*} \label{cot-a}
\cot(x,y,a) = \frac{a x_- x_+ + a^{-1} y_- y_+}{y_- x_+ - y_+ x_-}
\end{equation*}
so
\begin{equation*}
\cot(x,y,a) = \cot(x,y,1) \cdot \frac{a x_- x_+ + a^{-1} y_- y_+}{x_- x_+ + y_- y_+}
%  \cot(x,y,1) \frac{a (\cos(\beta+\alpha)-\cos(\beta))(\cos(\beta-\alpha)-\cos(\beta)) + a^{-1} (\sin(\beta+\alpha)-\sin(\beta))(\sin(\beta-\alpha)-\sin(\beta))}{...}
\end{equation*}

\noindent Note that:

\begin{align*}
x_- x_+ &= (x (\cos \alpha - 1) + y \sin \alpha) (x (\cos \alpha - 1) - y \sin \alpha) \\&= x^2 (\cos \alpha - 1)^2 - y^2 \sin^2 \alpha \\
y_- y_+ &= \dots                                                   = y^2 (\cos \alpha - 1)^2 - x^2 \sin^2 \alpha \\
x_- x_+ + y_- y_+                                                 &= (\cos \alpha - 1)^2 - \sin^2 \alpha 
\end{align*}

\noindent and since $x=\cos{\beta}$, $y=\sin{\beta}$, 
$x^2 + y^2 = \cos^2{\beta} + \sin^2{\beta} = 1$ (we use a unit circle by convention)
and $\sin^2 \alpha + \cos^2 \alpha = 1$ we get:

\begin{multline}
\label{eqn:3}
\int_\beta \cot(\beta,a)
= \frac{\int_\beta \cdot ( a (A \cos^2 \beta - B \sin^2 \beta) + a^{-1} (A \sin^2 \beta - B \cos^2 \beta))}{A-B} 
\\
= \left(\int_\beta \cos^2 \beta \right) \cdot \frac{a A-a^{-1}B}{A-B}
+ \left(\int_\beta \sin^2 \beta \right) \cdot \frac{a^{-1}A-a B}{A-B}
\end{multline}

\noindent where $\int_\beta$ is any linear functional on functions of $\beta$ such that $\int_\beta 1 = 1$,
$A = (\cos \alpha - 1)^2$ and $B = \sin^2 \alpha$ are constant and don't depend on $\beta$.
Note that $\cot(\beta,1)$ is a constant function:
it depends on $\alpha$, but not on $\beta$,
and can be easily computed from geometry:
\[ \int_\beta \cot(\beta,1) = \cot(\cdot,1) = -\cot(\alpha) \]

Under the assumption
\begin{equation}
 \label{fourier-2}
 \int_\beta  \exp(\pm2i\cdot\beta) = 0
\end{equation}
using
\[\cos(\beta) = \frac{\exp(i\beta) + \exp(-i\beta)}2 \]
we have
\[ \cos(\beta)^2 = \frac12 + \frac14\cdot{(\exp(i\cdot2\beta) + \exp(-i\cdot2\beta))} \]
%Em jeito similar, sin(\beta) = (exp(i\beta) - exp(-ib))/(2i) e sin(\beta)^2 = 1/2 - (exp(i 2\beta) + exp(-i 2\beta))/4. [usando sin^2 = 1 - cos^2 esta linha não é necessária]
which implies
\[
\sin^2 \beta = \int_\beta \cos^2 \beta = \frac12,
\]
using the above in \Cref{eqn:3}, obtain:
\begin{equation*}
\frac{\int_\beta \cot(\beta,a)}{\int_\beta \cot(\beta,1)}
= \frac12 \cdot \frac{aA-a^{-1}B}{A-B}
+ \frac12 \cdot \frac{a^{-1}A-a B}{A-B}
= \frac{a+a^{-1}}2
\end{equation*}

So \cref{fourier-2} implies that the ``integral'' $\int_\beta \cot(\beta,a)$
equals $\frac{a+a^{-1}}{2}$ times  $\int_\beta \cot(\beta,1)$, and the latter factor equals $-\cot \alpha$.

To summarize the above:

\begin{equation} \label{space-average}
\int_\beta  \cot(\beta;a,\alpha)  = - \frac{a+a^{-1}}{2} \cdot \cot(\alpha)
\end{equation}
for any linear normalized functional $\int_\beta$ that satisfies \cref{fourier-2}.

If the integral $\int_\beta f(\beta)$ equals to the integral with respect to the normalized Lebesgue measure
(i.e., the so-called spatial average $\frac1{2\pi} \int_{\beta=0}^{2\pi} f(\beta) d\beta$),
then \Cref{fourier-2} 
follows from either $\cos(\beta)=\sin(\beta+\pi/2)$ or Cauchy's residue formula.

On the other hand for fixed $N$ and any $\gamma$ we can consider the discrete integral

\[ \int_\beta f = \sum_{k=0}^{N-1} f(\gamma + \frac{2\pi i k}{N}) \]

For any $\gamma$ and integer $M$
\[ \sum_{k=0}^{N-1} \exp(i M \cdot (\gamma +\frac{2\pi k}{N}))
= \exp(i M \gamma) \cdot \sum_{k=0}^{N-1} \exp(2\pi i\frac{k M}{N}). \]
Number $\sum_{k=0}^{N-1} \exp(2\pi i\frac{k M}{N})$ vanishes unless $M$ is divisible by $N$,
in which case it equals $N$. 
This is known as the orthogonality of characters (for cyclic group). Explicitly:
\[ 
\sum_{k=0}^{N-1} \exp\big(2\pi i\cdot \frac{kM}{N}\big) 
= \frac{1-\exp(2\pi i\cdot MN)}{1-\exp(2\pi i\cdot\frac{M}{N})} 
= \frac0{1-\exp(2\pi i\cdot\frac{M}{N})} = 0 
\]
%
%[ Se N=1 ou N=2 o denominador é 0, e na verdade a soma não é zero ]
%
This implies that if $N>2$, the time average of $\cos^2(\beta)$ equals the spatial average.
Hence for $N>2$ periodic time averages of $\cot\theta$ are equal to their spatial time averages, and are invariant.
